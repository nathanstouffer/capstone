\section{Methodology}
\label{sec:methodology}

% provide enough information so that someone can replicate

Much of this project is devising a method to bootstrap blockchain nodes.
Since this is a conceptual task, it cannot be replicated.
However, the experimental verification of our protocol can be replicated.
Section \ref{sec:methodology} is devoted to providing the necessary information for someone to replicate our implementation.

% network simulator with dht

In order to match our model as closely as possible with our implementation, we chose to implement our solution on a network simulator using a full-fledged DHT.
This is not strictly necessary but we felt that it made our results more applicable to a real implementation of our idea.

% digitally signed parents

A neglected detail of the solution is the process of selecting parents in the tangle.
The tangle manager has complete control over \textit{which} votes are selected as parents, but we require that the parents are digitally signed in the vote.
This means that the manager must tell the voter which votes are the parents and the voter includes that information in their vote.
While this is some overhead communication for our solution, it prevents a malicious actor from changing the structure of a tangle to their advantage.

% attacker intelligence

The most important aspect of the implementation is the intelligence of the attackers.
We expect the constraints of our solution to prevent an attacker from \textit{successfully} fooling a bootstrapping node.
But the only way we can verify this is to simulate intelligent attackers.
We have divided attacks into three categories: creating votes, deleting votes, and reordering votes.
An accurate replication of this project should at least implement these strategies.
Implementing more strategies is welcome, since it provides more information on our system!

% chi-squared test

We used the $\chi ^2$ test as follows.
Consider some sample of votes $S$ and our expected proportion of agreeing votes $\beta$.
Our distribution comes in the form $(majority, other)$ where $majority$ references the number of valid votes that supported the most popular state and $other$ is the number of valid votes that did not support the most popular state.
We have an expected distribution $(\beta|S|, (1-\beta)|S|)$ and an actual distribution of $(S_{maj}, S_{oth})$.
Then we can use the $\chi ^2$ test to determine the probability that our sample comes from the expected distribution.

\subsection{Design}

The design pattern that appears in this project is the Strategy Pattern.
The design pattern presents itself in the different attack styles that we will simulate: creating votes, deleting votes, and reordering votes.
We will have an adversary interface with a method callled \textit{attack()}.
Each class that implements an attack type will implement that interface.
Figure \ref{fig:strategy} shows a visual of our Unified Modeling Language diagram for this pattern.

\newcommand{\operation}{$+$\textit{sortedVotes()} : String[]}

\begin{center}
	\vspace{1em}
    \begin{tikzpicture}[scale=0.9]
    % uml figure

    % boxes (w,l) = (5,3)
    \draw [very thick] (6,6) rectangle (11,9);
    \node at (8.5,8.5) {IVoteSorter};
    \draw [thick] (6,8) -- (11,8);
    \node at (8.5,7.65) {\operation};

    \draw [very thick] (0,0) rectangle (5,3);
    \node at (2.5,2.5) {RndSorter};
    \draw [thick] (0,2) -- (5,2);
    \node at (2.5,1.65) {\operation};

    \draw [very thick] (6,0) rectangle (11,3);
    \node at (8.5,2.5) {BFSSorter};
    \draw [thick] (6,2) -- (11,2);
    \node at (8.5,1.65) {\operation};

    \draw [very thick] (12,0) rectangle (17,3);
    \node at (14.5,2.5) {DescSorter};
    \draw [thick] (12,2) -- (17,2);
    \node at (14.5,1.65) {\operation};

    % lines
    \draw [dashed, -{Triangle[open, width=3.5mm, length=3mm]}] (8.5,3) to (8.5,6);
    %\draw [dashed, very thick] (6.5,3) -- (6.5,5.7);
    \draw [dashed, very thick] (2.5,3) -- (2.5,4);
    \draw [dashed, very thick] (2.5,4) -- (6.5,4);
    \draw [dashed, very thick] (14.5,3) -- (14.5,4);
    \draw [dashed, very thick] (14.5,4) -- (6.5,4);

    \end{tikzpicture}
    \captionof{figure}{Strategy Pattern for vote sorters\label{fig:strategy}}
\end{center}

