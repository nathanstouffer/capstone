\section{Introduction}
\label{sec:introduction}

% >> This needs to tell a story: big picture problem, identify a "shortcoming", narrow it down to a specific gap that you are going to address.

% societal problem
% technical problem
% technical solution
% expected impact

Blockchain was introduced in order to provide distributed consensus without relying on a central party~\cite{nakamoto2009Bitcoin}.
Avoiding a central party provides two benefits.
First, there is no central location that an attacker can access to compromise a system.
A centralized blockchain is much easier to compromise than a decentralized blockchain.
Second, there is no single party with complete control over a system.
An example of where decentralized control is useful is in international currency exchange.
Banks have the power to charge high transaction fees for international monetary transfers and customers are forced to pay the fees because there is no alternative.
Legitimate cryptocurrencies have enough competition between parties (miners) that there is no room for a monopoly, giving customers a cheaper option.

Bitcoin was released to the general public in 2009 as the first implementation of a blockchain.
Bitcoin garnered a lot of support and quickly became one of the most prominent cryptocurrencies.
Since Bitcoin's inception, blockchains have found other applications.
They are used in other cryptocurrencies, verify supply chains, and increase device autonomy in the Internet of Things~\cite{cai2018DecentralizedApplications}.
Blockchains are even being used to provide security and privacy in the medical field~\cite{siyal2019MedicalApplications}.
Blockchain technology has the potential to enhance data security in a wide range of diverse applications.

Blockchain’s services are best realized when control of the chain is decentralized.
Decentralized control can become difficult to achieve when blockchains grow to an unmanageable size.
I am writing this proposal to request funding to work on a method to truncate  blockchains.
My work on this project will be a continuation from the REU research I collaborated on this summer.
We have a solution sketch and a prototype implementation, but more work needs to be done to prove the correctness of the solution and to translate the prototype to a practical implementation.
If successful, this project has the possibility of making blockchains more effective, extending data reliability and user privacy~\cite{cai2018DecentralizedApplications}~\cite{siyal2019MedicalApplications}.

The remaining sections provide information regarding background information, my plan, an estimated time line, and other logistical details.
