\section{Appendix}
\label{sec:appendix}

Here is the source file for the primary data structure: the Tangle.
If you would like to see the rest of the code base, I can give you access to the repository.
I chose not to include more files because I think the source code amounts to about 50 pages of text.

\begin{Verbatim}

package tangly;

import java.io.*;
import java.util.Hashtable;
import java.util.ArrayList;
import java.util.Set;
import java.util.Arrays;

import tangly.util.*;
import static tangly.util.Functions.*;

public class Tangle implements Serializable {

// STATIC VARIABLES

private static String ROOT0 = 
        bytesToHexString(sha256("00000".getBytes()));	
private static String ROOT1 = 
        bytesToHexString(sha256("11111".getBytes()));

// CLASS VARIABLES

private String tangle_loc;
private Hashtable<String, Vertex> ht;
private ArrayList<String> tips;

// CONSTRUCTOR

public Tangle(String tangle_loc) {
 this.tangle_loc = tangle_loc;
 this.ht = new Hashtable<String, Vertex>();		
 this.tips = new ArrayList<String>();	
 this.tips.add(Tangle.ROOT0);
 this.tips.add(Tangle.ROOT1);
}

// static method to return a deserialized version of the tangle
// should be used by someone who has a serialized version of a tangle
// and needs to instantiate tangle vote in memory. This could be used
// by a DHT node recieving a tangle from another node or a 
// deciding node recieving tangles from the DHT
public static Tangle deserialize(byte[] serialized) {
 try {
  ByteArrayInputStream bis = new ByteArrayInputStream(serialized);
  ObjectInputStream ois = new ObjectInputStream(bis);
  Tangle recon = (Tangle) ois.readObject();
  return recon;
 } catch(IOException e){ 
  System.err.println("caught exception: " + e);
 } catch(ClassNotFoundException e){
  System.err.println("caught exception: " + e);
 }
 return null;
}

// PUBLIC METHODS

// method to serialize tangle for transmission
public byte[] serialize() {
 try{
  ByteArrayOutputStream bos = new ByteArrayOutputStream();
  ObjectOutputStream oos = new ObjectOutputStream(bos);
  oos.writeObject(this);
  oos.close();
  return bos.toByteArray();
 } catch(FileNotFoundException e){
  System.err.println("caught exception: " + e);
 } catch(IOException e){
  System.err.println("caught exception: " + e);
 }
 return new byte[0];
}

// method to insert a vote into the tangle
public void insert(Vertex v) {
 this.ht.put(v.identifierH(), v);
 this.tips.add(v.identifierH());
 if (this.tips.size() > 2) { this.tips.remove(v.parent1H()); }
 if (this.tips.size() > 2) { this.tips.remove(v.parent2H()); }
}

// method to return the next two tips that should be used
public String[] nextTips() {
 return new String[] { new String(this.tips.get(0)), 
                       new String(this.tips.get(1)) };
}

// method to mark the vote at key as invalid
public void markInvalid(String key) {
 if (this.contains(key)) { this.ht.get(key).vote().markInvalid(); }
 else { System.err.println("No such key in hash table"); }
}

// method to mark the vote at key as deleted
public void markDeleted(String key) {
 if (this.contains(key)) {
 this.ht.get(key).vote().markInvalid();
 this.ht.get(key).vote().markDeleted();
}
 else { System.err.println("No such key in hash table"); }
}

// method to return whether the desired vote is in the tangle
public boolean contains(String key) {
 if (key.equals(Tangle.ROOT0) || key.equals(Tangle.ROOT1)) { 
  return true; 
 }
 if (this.ht.containsKey(key)) {
  if (!this.ht.get(key).vote().isMarkedDeleted()) {
   return true;
  }
  else { return false; }
 }
 else { return false; }
}

// method to return whether the desired vote is in the tangle
public boolean contains(Vote vote) {
 return this.contains(vote.identifierH());
}

// method to return whether the desired vertex is in the tangle
public boolean contains(Vertex vert) {
 return this.contains(vert.identifierH());
}

// method to populate the children of each vote with the 
// reverse of the parents
public void populateChildren() {
 this.clearChildren();						
 for (String child : this.keySet()) {
  Vertex v    = this.ht.get(child);	
  String p1 = v.parent1H();			
  String p2 = v.parent2H();
  if (!p1.equals(Tangle.ROOT0) && !p1.equals(Tangle.ROOT1)) {	 
   this.ht.get(p1).addChild(child);					
  }
  if (!p2.equals(Tangle.ROOT0) && !p2.equals(Tangle.ROOT1)) {
   this.ht.get(p2).addChild(child);
  }
 }
}

// method to return the keys of all votes that have been marked as invalid
public ArrayList<String> invalidKeys() {
 ArrayList<String> invalid = new ArrayList<String>();
 for (String key : this.keySet()) {
  if (this.ht.get(key).vote().isMarkedInvalid()) { invalid.add(key); }
 }
 return invalid;
}

public ArrayList<String> validKeys() {
 ArrayList<String> valid = new ArrayList<String>();
 for (String key : this.keySet()) {
  if (!this.ht.get(key).vote().isMarkedInvalid()) { valid.add(key); }
 }
 return valid;
}

// PRIVATE METHODS

// method to clear the children of each vote
private void clearChildren() {
 Set<String> keys = this.keySet();
 for (String key : keys) { this.ht.get(key).clearChildren(); }
}

// method to return the number of votes in given round in the ancestry of the seeds
public int numVotesInRound(int num_rnds, int p_round, String[] seeds) {
 int counter = 0;
 Hashtable<String, Boolean> added = new Hashtable<String, Boolean>();
 ArrayList<String> queue = new ArrayList<String>();	
 for (int s = 0; s < seeds.length; s++) {
  queue.add(seeds[s]);
  added.put(seeds[s], true);
 }
 for (int q = 0; q < queue.size(); q++) {
  String key = queue.get(q);										
  if (!key.equals(Tangle.ROOT0) && !key.equals(Tangle.ROOT1)) {
   Vertex vert = this.ht.get(key);								
   int rnd = round(num_rnds, this.tangle_loc, vert.vote());	
   if (rnd == p_round) { counter++; }							
   if (!added.containsKey(vert.parent1H())) {					
    queue.add(vert.parent1H());
    added.put(vert.parent1H(), true);
   }
   if (!added.containsKey(vert.parent2H())) {					
    queue.add(vert.parent2H());
    added.put(vert.parent2H(), true);
   }
  }
 }
 return counter;
}

// method that returns whether there are enough votes in the
// ancestry of the seeds to satisfy the min_votes requirement
public boolean atLeastMinVotes(int num_rnds, int min_votes, 
                               int p_round, String[] seeds) {
 int counter = 0;
 Hashtable<String, Boolean> added = new Hashtable<String, Boolean>();
 ArrayList<String> queue = new ArrayList<String>();						
 for (int s = 0; s < seeds.length; s++) {
  queue.add(seeds[s]);
  added.put(seeds[s], true);
 }
 for (int q = 0; q < queue.size(); q++) {								
  String key = queue.get(q);										
  if (!key.equals(Tangle.ROOT0) && !key.equals(Tangle.ROOT1)) {
   Vertex vert = this.ht.get(key);								
   int rnd = round(num_rnds, this.tangle_loc, vert.vote());	
   if (rnd == p_round) { counter++; }							
   if (!added.containsKey(vert.parent1H())) {					
    queue.add(vert.parent1H());
    added.put(vert.parent1H(), true);
   }
   if (!added.containsKey(vert.parent2H())) {					
    queue.add(vert.parent2H()); 
    added.put(vert.parent2H(), true);
   }
  }
  if (counter >= min_votes) { return true; }
 }
 return (counter >= min_votes);
}

// GETTER METHODS

public Set<String> keySet()   { return this.ht.keySet(); }
public Vertex get(String key) { return this.ht.get(key); }
public String getTangleLoc()  { return this.tangle_loc; }

// STRING METHODS

public String toString() {
 String ret = new String(this.tangle_loc + "\n");
 for (String key : this.keySet()) {
  ret += this.ht.get(key).toString();
 }
 return ret;
}

}



\end{Verbatim}
