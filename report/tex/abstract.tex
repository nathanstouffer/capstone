\begin{abstract}
    % blockchains are interesting and useful
    Blockchains have applications in cryptocurrencies, supply chain tracking, and providing data integrity.
    A blockchain provides an immutable, decentralized record.
    To provide immutability, some blockchains use Proof of Work to construct an ever growing chain of blocks in a peer-to-peer network.
    Participants, known as miners, expend computational resources to create new blocks.
    By design, users trust the longest blockchain and blocks are intentionally difficult to create.
    As a result, data far back in the blockchain is immutable.

    % problem with blockchains
    By design, miners have collective, but not individual, control over a blockchain.
    When miners are sufficiently decentralized, it is infeasible for a coalition of miners to gain explicit control of a blockchain.
    Explicit control would allow a coalition to decide which blocks are added to the chain, reducing the integrity of the system.
    Thus a blockchain performs better when control of the chain is decentralized.
    % maybe get rid of this part
    Decentralization can be difficult to achieve when blockchains grow to an unmanageable length.
    To begin mining, one must download and process the entire chain (a process called bootstrapping).
    This is not a problem when the chain is relatively small, but a larger chain (such as Bitcoin) can take days to download and process.
	Such a long chain prevents lightweight devices from becoming miners and incredibly long bootstrapping time deters participation from many users who do have sufficient space.
    Together, these issues make joining a long blockchain more difficult.

    % idea for a solution
    Over the summer of 2020, I worked with collaborators to devise a faster bootstrapping method.
    At a high level, our solution has miners of recent blocks vote for a blockchain summary.
    If sufficiently many votes agree, bootstrapping nodes can trust the blockchain summary and no longer need to download the entire chain.
    This drastically reduces bootstrapping time.
    For my capstone project, I continued the solution from the previous summer and implemented a model of the solution.
    Using the model, I ran simulations and extracted experimental results.
\end{abstract}
\newpage
