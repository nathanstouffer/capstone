\section{Introduction}
\label{sec:introduction}

% >> This needs to tell a story: big picture problem, identify a "shortcoming", narrow it down to a specific gap that you are going to address.

% societal problem
% technical problem
% technical solution
% expected impact

Blockchain was introduced as a way to provide distributed consensus without relying on a central party~\cite{nakamoto2009Bitcoin}.
Bitcoin was released to the general public in 2009 as the first implementation of a blockchain.
Bitcoin quickly garnered a lot of support and became one of the most prominent cryptocurrencies.
Since Bitcoin's inception, blockchains have found many applications.
They are used in other cryptocurrencies, verify supply chains, and increase device autonomy in the Internet of Things~\cite{cai2018DecentralizedApplications}.
Blockchains are even being used to provide security and privacy in the medical field~\cite{siyal2019MedicalApplications}.
Blockchain technology has the potential to enhance data security in a wide range of diverse applications.

Blockchains are decentralized.
Avoiding a central party provides two benefits.
First, there is no central location for an attacker to compromise, giving blockchain a robustness that centralized systems struggle to achieve.
This is a nice property because no centralized attack (such as attempting to flood a server with requests) can succeed against a blockchain; it is just too large.
Second, there is no single party with complete control over a system, preventing a monopoly of power.
An example of where lacking a central party is useful is international currency exchange.
Banks have the power to charge high transaction fees for international exchanges and customers are forced to pay the fees because there is no alternative.
If one transfers money over a cryptocurrency, there is too much competition between parties (miners) for a monopoly to emerge.
Since there is no monopoly, the market finds a fair equilibrium between miners and users.

Blockchain’s services are best realized when control of the chain is decentralized.
Decentralized control can become difficult to achieve when blockchains grow to an unmanageable size.
Joining a well-established blockchain can take days, excluding lightweight devices and reducing the likelihood a large device joins.
This means a blockchain is less decentralized and prevents Internet of Things devices from participating in a blockchain network.
This project provides and analyzes a new, off-chain, efficient, and secure method for joining a blockchain.
This project contributes towards making blockchains more effective and increasing their decentralization.
More effective blockchains can extend data reliability and user privacy~\cite{cai2018DecentralizedApplications}~\cite{siyal2019MedicalApplications}.

The remaining sections provide background information, our solution, methodology, results, and a discussion.
