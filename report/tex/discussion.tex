\section{Discussion}
\label{sec:discussion}

% complement results with opinions and reasoning

The time and space requirements for our protocol are significantly less than current bootstrapping systems.
As previously noted, joining Bitcoin can take multiple days whereas our voting system has the possibility of verifying a state within 1000 seconds (although we should recall that this is only a lower bound).
We should note that this is not the bootstrapping time, just the time required to verify a blockchain state.
However, the verified state eliminates the need for any blocks prior to the state is attached to, drastically reducing time and space requirements.

The effectiveness of the deleting strategy ran as expected.
The graphs in Figure \ref{fig:deleter} show that, as long as we require a minimum number of votes to be valid, we do not have to worry about dishonest managers deleting votes since they will not be able to affect the ratio.

The rejecting strategy is much more concerning.
It was expected that rejecting would be more effective than deleting because deleting is such a naive strategy.
A 25\% gain (seen in Figure \ref{fig:rejecter}) gives too much power to an attacker.
We do believe that this obstacle can be overcome.
There are a lot of system parameters involved in the simulation and we just used fixed values for these experiments.
We suspect that increasing the threshold $t$ for the number of votes in the previous round combats the ability of the rejecter to gain so much power.
On the other hand, it also means the system is less likely to reach consensus because $t$ might large enough to stop tangle progress altogether.
Regardless, messing around with the system parameters is necessary before the true effectiveness of the protocol can be established.